\section{Аналитическая часть}

В данном разделе содержится описание предметной области, проводится рассмотрение основных способов симуляции движения жидкостей.

\subsection{Проблема моделирования течения жидкостей}

% Таблица сравнения методов расчета течения жидкостей
\begin{table}[htb]
    \centering
    \caption{Таблица сравнения методов расчета течения жидкостей}
    \label{tab:methods}
    \begin{tabular}{|l|l|l|l|}
        \hline
        \textbf{Метод} & \textbf{Размерность} & \textbf{Число узлов} & \textbf{Число шагов} \\
        \hline
        \textbf{Равномерный} & \textbf{По производной} & \textbf{По производной} & \textbf{По производной} \\
        \hline
        \textbf{Равномерный} & \textbf{По производной} & \textbf{По производной} & \textbf{По производной} \\
        \hline
        \textbf{Равномерный} & \textbf{По производной} & \textbf{По производной} & \textbf{По производной} \\
        \hline
        \textbf{Равномерный} & \textbf{По производной} & \textbf{По производной} & \textbf{По производной} \\
        \hline
    \end{tabular}
\end{table}

\subsection{Уравнения Навье-Стокса}

\subsection{Сеточный метод Эйлера}

\subsection{Метод Лагранжа}

\pagebreak